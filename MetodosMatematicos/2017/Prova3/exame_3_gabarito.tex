\documentclass[12pt]{article}


%\usepackage{mathpazo} 
%\usepackage{millennial} % nao funciona
%\usepackage[math]{anttor}
\usepackage{fouriernc}

\usepackage[latin1]{inputenc}
\usepackage[portuguese]{babel}
\usepackage[T1]{fontenc}
\usepackage{amsmath}
\usepackage{amsfonts}

%pacotes para fazer letra cursiva
\usepackage{mathrsfs}


% possibilida criar figuras 1a) e 1b), por exemplo
\usepackage{subfigure}
\usepackage[pdftex]{graphicx}



%para criar o espacamento usual no primeiro paragrafo
%\usepackage{indentfirst}

% para criar os links no texto quando cita equacoes, figuras, referencias, secoes, etc...
\usepackage{hyperref}

% para colorir os links criados anteriormente
\hypersetup{ colorlinks,
linkcolor=blue,
filecolor=darkgreen,
urlcolor=blue,
citecolor=blue }
\usepackage{microtype}
\usepackage[nottoc]{tocbibind} 
\usepackage{graphicx}
\usepackage{graphicx,color}

\usepackage{amssymb}
\usepackage{amsthm}


% altera a fonte nas legendas das figuras
\usepackage[font=small,format=plain,labelfont=bf,up,textfont=it,up]{caption}

\usepackage{geometry}
 \geometry{
 a4paper,
 total={210mm,297mm},
 left=20mm,
 right=20mm,
 top=20mm,
 bottom=20mm,
 }


\setlength{\parindent}{0em} %altera o espacamento do paragrafo
\setlength{\parskip}{1em}  %altera o espaco entre dois paragrafos
\renewcommand{\baselinestretch}{1.1} %altera o espacamento entre as linhas

\begin{document}


\subsection*{Jata�, August 31st, 2017. UFG, Jata�. Prof. Paulo Freitas Gomes.}

\subsection*{Mathematical Methodos for Physicists. Exam 2.}


\vspace{0.4 in}
\noindent
%Aluno: \_\_\_\_\_\_\_\_\_\_\_\_\_\_\_\_\_\_\_\_\_\_\_\_\_\_\_\_\_\_\_\_\_\_\_\_\_ Matr�cula:\_\_\_\_\_\_\_\_\_\_
\textbf{Name:} \noindent\rule{12cm}{0.5pt}. 



{\color{red}\textbf{General instructions:}} \textit{Be organized.} Write your name in all pages you use and put numbers on them. Indicate explicitly all general results you use. Show the physical reasoning and be as clear as possible. Remember: someone has to understand what you write. Everything you need to solve this exam is in the 3 indicated books: \textit{Mathematical Methods} from Arken, \textit{Electrodynamics} and \textit{Introduction to Quantum Mechanics} from Griffiths.



\begin{center}
\noindent\rule{13cm}{0.5pt}
\end{center}

{\color{blue}\textbf{1)}} An atomic (quantum mechanical) particle is confined inside a rectangular box of sides $a$, $b$, and $c$. The particle is described by a wave function $\Psi$ that satisfies the Schr�dinger wave equation
{\color{blue}
\begin{equation} 
-\dfrac{\hbar^2}{2m} \nabla^2 \Psi = E \Psi. \nonumber
\end{equation}}
The wave function is required to vanish at each surface of the box (but not to be identically zero). This condition imposes constraints on the
separation constants and therefore on the energy $E$. Show that the smallest value of $E$ for which such a solution can be obtained is:
{\color{blue}
\begin{equation}
E = \dfrac{\pi^2 \hbar^2}{2m} \left( \dfrac{1}{a^2} + \dfrac{1}{b^2} + \dfrac{1}{c^2} \right). \nonumber
\end{equation}}

\begin{center}
\noindent\rule{13cm}{0.5pt}
\end{center}

{\color{blue}\textbf{2)}} An infinitely long rectangular metal pipe (sides $a$ and $b$) is grounded, but one end at $x=0$ is maintained at a specified potential $V_0(y,z)$, as indicated in figure \ref{tubo_metalico_3d}. a) What are the boundary conditions? \textit{Hint}: see figure \ref{tubo_metalico_3d}. b) Show that the solution and the coefficients are:
{\color{blue}
\begin{eqnarray}
V(\textbf{r}) &=& \sum_{n=1}^{\infty} \sum_{m=1}^{\infty} C_{n,m} \sin (n\pi y/a) \sin (m\pi z/b) \exp \left(-\pi  x \sqrt{(n/a)^2 + (m/b)^2} \right), \nonumber \\
C_{n,m} &=& \dfrac{4}{ab} \int_0^a \int_0^b V_0(y,z) \sin (n\pi y/a)dy \sin (m\pi z/b). \nonumber
\end{eqnarray}}
c) Show that, if $V_0(y,z)=V_0$ is a constant, the potential inside the pipe is:
{\color{blue}
\begin{equation}
V(x,y,z) = \dfrac{16V_0}{\pi^2} \sum_{n,m=1,3,5,...}^{\infty} \dfrac{1}{nm} \exp \left( -\pi x \sqrt{ \dfrac{n^2}{a^2} + \dfrac{m^2}{b^2} } \right) \sin (n\pi y/a) \sin (m \pi z /b). \nonumber
\end{equation}}



\begin{figure}[!h]
\centering
\subfigure{a)
\includegraphics[width=2.5 in]{figuras/tubo_metalico_3d.png}
\label{tubo_metalico_3d} 
}
\subfigure{b)
\includegraphics[width=1.7 in]{figuras/orbita.pdf}
\label{orbita}
} 
\caption{\subref{tubo_metalico_3d} Infinitely long rectangular metal pipe of problem 2). \subref{orbita} Electron and proton in the hydrogen atom.}
\end{figure}


\begin{center}
\noindent\rule{13cm}{0.5pt}
\end{center}


{\color{blue}\textbf{3)}} \textbf{Hydrogen Atom}. Let's tackle the quantum mechanical description of the Hydrogen Atom. Consider the proton to be motionless and the electron to be orbiting around the proton, depicted in figure \ref{orbita}. The Coulomb interaction between then is:
{\color{blue}
\begin{equation}
V(r) = -\dfrac{e^2}{4\pi \epsilon_0 r}
\end{equation}}
where $e$ is the electron electric charge, $\epsilon$ is the electric permittivity of free space and $r$ is the distance between them. The time-independent Schr�dinger equation is:
{\color{blue}
\begin{equation}
-\dfrac{\hbar^2}{2m} \nabla^2 \Psi (\textbf{r}) + V\Psi (\textbf{r}) = E \Psi (\textbf{r}). \nonumber
\end{equation}}
Suppose separation of variables in spherical coordinates $(r,\theta,\phi)$, so $\Psi (\textbf{r}) = R(r) Y(\theta, \phi)$. Show that the solution of the angular part is the spherical harmonics:
{\color{blue}
\begin{equation}
Y_l^m (\theta,\phi) = \epsilon \sqrt{\dfrac{2l+1}{4\pi} \dfrac{(l-\vert m \vert )!  }{ (l+\vert m \vert )! }  } e^{im\phi} P_l^m (\cos \theta). \nonumber
\end{equation}}

b) Show that the radial equation is:
{\color{blue}
\begin{equation}
-\dfrac{\hbar^2}{2m} \dfrac{d^2u}{dr^2} + \left[  \dfrac{\hbar^2}{2m} \dfrac{l(l+1)}{r^2} -\dfrac{e^2}{4\pi \epsilon_0 r} \right] u = E u. \nonumber
\end{equation}}

c) The general solution is:
{\color{blue}
\begin{equation}
\Psi_{nlm} = e^{-r/na} \left( \dfrac{2r}{na} \right)^l \sqrt{ \left( \dfrac{2}{na} \right)^3   \dfrac{(n-l-1)!}{2n \left[ (n+l)! \right]^3 }  } \left[ L_{n-l-1}^{2l+1} (2r/na) \right] Y_l^m ( \theta, \phi). \nonumber
\end{equation}}
where:
{\color{blue}
\begin{equation}
L_q^p(x) = (-1)^p \left( \dfrac{d}{dx} \right)^p L_q (x), \qquad \qquad L_q(x) = e^x \left( \dfrac{d}{dx} \right)^q (e^{-x}x^q),  
\end{equation}}
are the associated Laguerre polynomial and the Laguerre polynomial. Show that the ground state and the first excited wave functions are:
{\color{blue}
\begin{equation}
\Psi_{100} (r,\theta,\phi) = \dfrac{e^{-r/a}}{a\sqrt{\pi a}}, \qquad \qquad \Psi_{200} (r,\theta,\phi) = \left( 2-\dfrac{r}{a} \right)   \dfrac{e^{-r/2a}}{4a\sqrt{2\pi a}},   \nonumber
\end{equation}}
where $a = (4\pi \epsilon_0 \hbar^2)/(me^2)$ is the so-called Bohr radius. \textit{Hint}: define properly the needed functions: spherical harmonics, associated Laguerre polynomial and Laguerre polynomial.


\begin{center}
\noindent\rule{13cm}{0.5pt}
\end{center}


\begin{flushright}
\begin{small}
\textit{ ... if you're willing to go through all the battling to get where you want to be, who's got the right to stop you? Maybe you got something you never finished, something you really want to do, something you never said to anyone, and you're told no, even after you paid your dues? Who's got the right to tell you no, who? Nobody! It's your right to listen to your gut, because you have the right to be where you want to be and do whatever you want to do!
}  
\textbf{Rocky Balboa}
\end{small}
\end{flushright}


\begin{center}
\noindent\rule{15cm}{0.5pt}
\end{center}


\newpage

{\color{red}Expected solutions.}

{\color{red}1). Value: 3.0} Let's put one vertex of the box in the origin $(0,0,0)$ and the box itself in the first octant. In this geometry and using cartesian coordinates the boundary conditions are:
\begin{eqnarray}
\Psi(x=0) = \Psi(y = 0) = \Psi(z=0)= 0, \nonumber \\
\Psi(x=a) = \Psi(y = b) = \Psi(z = c) = 0. \nonumber
\end{eqnarray}
In order to solve the differential equation we suppose separation of variables: $\Psi(\textbf{r}) = \varphi_1 (x) \varphi_2 (y) \varphi_3(z)$. The Schrodinger equation becomes:
\begin{equation}
-\dfrac{\hbar^2}{2m} \left[ \varphi_2 \varphi_3 \dfrac{d^2 \varphi_1}{dx^2} + \varphi_1 \varphi_3 \dfrac{d^2 \varphi_2}{dy^2} + \varphi_2 \varphi_2 \dfrac{d^2 \varphi_3}{dz^2} \right] =  E \varphi_1 \varphi_2  \varphi_3. \nonumber
\end{equation}
Now we have to convert this equation in 3 ordinary different equations. So we make $E= E_x + E_y +E_z$ and divide this equation by $\Psi = \varphi_1 \varphi_2 \varphi_3$:
\begin{equation}
-\dfrac{\hbar^2}{2m} \left[  \dfrac{1}{\varphi_1 } \dfrac{d^2 \varphi_1}{dx^2} + \dfrac{1}{\varphi_2 } \dfrac{d^2 \varphi_2}{dy^2} + \dfrac{1}{\varphi_3 } \dfrac{d^2 \varphi_3}{dz^2} \right] =  E_x + E_y +E_z. \nonumber
\end{equation}
We rewrite this equation as 3 equations:
\begin{equation}
\dfrac{d^2 \varphi_1}{dx^2} = - \varphi_1 k_x^2, \qquad \dfrac{d^2 \varphi_2}{dy^2} = - \varphi_2 k_y^2, \qquad \dfrac{d^2 \varphi_3}{dz^2} = - \varphi_3 k_z^2, \nonumber 
\end{equation}
where:
\begin{equation}
k_x^2 = \dfrac{2mE_x}{\hbar^2}, \qquad k_y^2 = \dfrac{2mE_y}{\hbar^2}, \qquad k_z^2 = \dfrac{2mE_z}{\hbar^2}. \label{alksjdf}
\end{equation}
The solutions will be:
\begin{eqnarray}
\varphi_1(x) &=& A_1 \cos k_x x + B_1 \sin k_x x, \nonumber \\
\varphi_2(y) &=& A_2 \cos k_y y + B_2 \sin k_y y, \nonumber \\
\varphi_3(z) &=& A_3 \cos k_z z + B_3 \sin k_z z. \nonumber 
\end{eqnarray}
We should now impose the boundary conditions. We begin with:
\begin{eqnarray}
\Psi(x=0) &=& 0  \qquad \therefore \qquad \varphi_1(0) = A_1 = 0, \nonumber \\
\Psi(y=0) &=& 0  \qquad \therefore \qquad \varphi_2(0) = A_2 = 0, \nonumber \\
\Psi(z=0) &=& 0  \qquad \therefore \qquad \varphi_3(0) = A_3 = 0. \nonumber 
\end{eqnarray}
The other conditions are:
\begin{eqnarray}
\Psi (x = a) &=& 0 \qquad \therefore \qquad \varphi_1(a) = \sin k_x a = 0, \nonumber \\
\Psi (y = b) &=& 0 \qquad \therefore \qquad \varphi_2(b) = \sin k_y b = 0, \nonumber \\
\Psi (z = c) &=& 0 \qquad \therefore \qquad \varphi_3(c) = \sin k_z c = 0. \nonumber 
\end{eqnarray}
which implies:
\begin{eqnarray}
k_x = \dfrac{n_1 \pi}{a}, \quad k_y = \dfrac{n_2 \pi}{b}, \quad k_z = \dfrac{n_3 \pi}{c}, \qquad n_1, n_2, n_3 = 1,2,3,...
\end{eqnarray}
Now we can obtain the energies from Eq. \ref{alksjdf}:
\begin{equation}
E_x = \dfrac{\hbar^2 k_x^2}{2m}, \qquad E_y = \dfrac{\hbar^2 k_x^2}{2m}, \qquad E_z = \dfrac{\hbar^2 k_x^2}{2m}. \nonumber
\end{equation}
To recover the energy $E$ we use the values for each $k$:
\begin{equation}
E(n_1,n_2,n_3) = E_x+E_y+E_z= \dfrac{\hbar^2 }{2m} \left( k_x^2 + k_y^2 + k_z^2 \right) = \dfrac{\hbar^2\pi^2 }{2m} \left( \dfrac{n_1^2}{a^2}+ \dfrac{n_2^2 }{b^2}+ \dfrac{n_3^2}{c^2} \right). \nonumber
\end{equation}
The lowest energy corresponds to $n_1=n_2=n_3=1$:
\begin{equation}
E(1,1,1) =\dfrac{\hbar^2\pi^2 }{2m} \left( \dfrac{1}{a^2}+ \dfrac{1 }{b^2}+ \dfrac{1}{c^2} \right). \nonumber
\end{equation}
as requested.



\begin{center}
\noindent\rule{13cm}{0.5pt}
\end{center}

{\color{red}Question 2.} 

{\color{red}2a). Value: 4.0} 
This is a real 3D problem and the equation to be solved is the Laplacian one:
\begin{equation}
\nabla^2 V(x,y,z) = \dfrac{\partial^2 V}{\partial x^2} + \dfrac{\partial^2 V}{\partial y^2} + \dfrac{\partial^2 V}{\partial z^2} = 0. \nonumber
\end{equation}
From the Figure \ref{tubo_metalico_3d} the boundary conditions are:
\begin{itemize}
\item [1] $V=0$ for $y=0$ and $y=a$. 
\item [2] $V=0$ for $z=0$ and $z=b$. 
\item [3] $V \rightarrow 0 $ for $x \rightarrow \infty$.
\item [4] $V = V_0(y,z)$ for $x=0$. 
\end{itemize}
{\color{red}2b). Value:} 
Now we use variable separation: $V(x,y,z) = X(x)Y(y)Z(z)$. The Laplace equation becomes:
\begin{equation}
\dfrac{1}{X} \dfrac{d^2 X}{d x^2} = - \dfrac{1}{Y} \dfrac{d^2 Y}{d y^2} - \dfrac{1}{Z}\dfrac{d^2 Z}{d z^2}. \nonumber
\end{equation}
Making each function equal to a constant:
\begin{equation}
\dfrac{1}{X} \dfrac{d^2 X}{d x^2} = C_1, \qquad \dfrac{1}{Y} \dfrac{d^2 Y}{d y^2} = C_2, \qquad \dfrac{1}{Z}\dfrac{d^2 Z}{d z^2} = C_3, \qquad C_1+C_2+C_3=0. \nonumber
\end{equation}
We should choose the proper constants. As the potential goes to zero when $x \rightarrow \infty$ we do $C_1>0$ so the solutions for $x$ will be real exponentials. On the other way we make $C_2<0$ e $C_3<0$ so the solutions in $y$ and $z$ will be oscillatory and fulfill the boundary conditions. Thus:
\begin{equation}
C_2 = -k^2, \qquad \qquad C_3 = -l^2, \qquad \qquad C_1 = k^2 +l^2. \nonumber
\end{equation} 
The Laplace equation becomes 3 ordinary different equations:
\begin{equation}
\dfrac{1}{X} \dfrac{d^2 X}{d x^2} = k^2 + l^2, \qquad \dfrac{1}{Y} \dfrac{d^2 Y}{d y^2} = -k^2, \qquad \dfrac{1}{Z}\dfrac{d^2 Z}{d z^2} = -l^2, \qquad C_1+C_2+C_3=0, \nonumber
\end{equation}
with the solutions:
\begin{eqnarray}
X(x) &=& A \exp \left( x \sqrt{k^2 + l^2} \right) + B \exp \left( -x \sqrt{k^2 + l^2} \right), \nonumber \\
Y(y) &=& C \sin ky + D \cos ky, \qquad \qquad Z(z) = E \sin lz + F \cos lz.  \nonumber
\end{eqnarray}
Now we apply some boundary conditions: 
\begin{itemize}
\item [1] demands $D = 0$ and $k =n\pi/a$ with a positive integer $n$. 
\item [2] demands $F = 0$ and $l =m\pi/b$ with a positive integer $m$. 
\item [3] demands $A=0$.
\end{itemize}
The solution takes the form:
\begin{equation}
V_{n,m}(\textbf{r}) = X(x)Y(y)Z(z) = C \sin (n\pi y/a) \sin (m\pi z/b) \exp \left(-\pi  x \sqrt{(n/a)^2 + (m/b)^2} \right). \nonumber 
\end{equation}
As the Laplace is linear, any combination is solution. So the general solution is:
\begin{equation}
V(\textbf{r}) = \sum_{n=1}^{\infty} \sum_{m=1}^{\infty} C_{n,m} \sin (n\pi y/a) \sin (m\pi z/b) \exp \left(-\pi  x \sqrt{(n/a)^2 + (m/b)^2} \right). \label{slkjdfpppp}
\end{equation}
Only now we can apply the last boundary condition, [4]:
\begin{equation}
V(x=0,y,z) = \sum_{n=1}^{\infty} \sum_{m=1}^{\infty} C_{n,m} \sin (n\pi y/a) \sin (m\pi z/b) = V_0(y,z). \nonumber
\end{equation}
This equation can be satisfied because the $\sin$ function make a complete set. We multiply by $\sin (p\pi y/a) \sin (q \pi z/b)$ and integrate to obtain:
\begin{eqnarray}
 \sum_{n=1}^{\infty} \sum_{m=1}^{\infty} C_{n,m} \int_0^a \sin (n\pi y/a) \sin (p\pi y/a)dy \int_0^b \sin (m\pi z/b) \sin (q\pi z/b) \nonumber \\
  = \int_0^a \int_0^b V_0(y,z) \sin (p\pi y/a)dy \sin (q\pi z/b). \nonumber
\end{eqnarray}
The function $\sin$ make a complete set because of its integration property:
\begin{equation}
 \int_{x_0}^{x_0+2\pi} \sin mx \sin nx  dx  = \left\{
\begin{array}{rl}
 \pi \delta_{m,n}& \text{if } m \neq 0  \\
\\
 0 & \text{if } m = 0
\end{array} 
\right.  \nonumber \\
\end{equation}
So only the terms $n=p$ and $m=q$ are non-zero:
\begin{equation}
C_{p,q} =  \dfrac{4}{ab} \int_0^a \int_0^b V_0(y,z) \sin (p\pi y/a)dy \sin (q\pi z/b) . \label{lsksjksmmzzzz}
\end{equation}

{\color{red}2c). Value:} If $V_0(y,z)=V_0$ we have:
\begin{eqnarray}
C_{n,m} &=& \dfrac{4V_0}{ab} \int_0^a \sin(n\pi y/a)dy \int_0^a \sin(m\pi z/b)dz, \nonumber \\
&=& \left\{
\begin{array}{rl}
 0& \text{if } n \text{ or } m \text{ is even}  \\
\\
 \dfrac{16V_0}{\pi^2 nm} & \text{if } n \text{ and } m \text{ are odd}
\end{array} 
\right.  \nonumber \\
\end{eqnarray}


\begin{center}
\noindent\rule{13cm}{0.5pt}
\end{center}

{\color{red}3a). Value: 3.0}
To solve the Schrodinger equation with the electrostatic potential, we shall use spherical coordinates. The Schrodinger equation will be:
\begin{equation}
-\dfrac{\hbar^2}{2m} \left[ 
\dfrac{1}{r^2 }  \dfrac{\partial}{\partial r} \left( r^2 \dfrac{\partial \Psi}{\partial r} \right)  +  \dfrac{1}{r^2 \sin \theta}  \dfrac{\partial}{\partial \theta} \left( \sin \theta \dfrac{\partial \Psi}{\partial \theta} \right) + \dfrac{1}{r^2 \sin^2 \theta} \dfrac{\partial^2 \Psi}{\partial \varphi^2} \right] + V\Psi = E \Psi. \nonumber
\end{equation}
We look for solutions of the type: $\Psi (\textbf{r}) = R(r) Y (\theta,\varphi)$. We put this solution in the Schrodinger equation, divide by $\Psi$ and multiply by $-2mr^2/\hbar^2$. After this we get:
\begin{equation}
\left\lbrace  \dfrac{1}{R }  \dfrac{d}{d r} \left( r^2 \dfrac{d R}{d r} \right) -\dfrac{2mr^2}{\hbar^2} \left[ V(r) - E \right] \right\rbrace +
\dfrac{1}{Y} \left\lbrace   \dfrac{1}{ \sin \theta}  \dfrac{d}{d \theta} \left( \sin \theta \dfrac{d Y}{d \theta} \right) + \dfrac{1}{\sin^2 \theta} \dfrac{d^2 Y}{d \varphi^2} \right\rbrace =0. \nonumber
\end{equation}
Each term in curly brackets depends on different variables. Making both of them equal to $l(l+1)$:
\begin{eqnarray}
\left\lbrace  \dfrac{1}{R }  \dfrac{d}{d r} \left( r^2 \dfrac{d R}{d r} \right) -\dfrac{2mr^2}{\hbar^2} \left[ V(r) - E \right] \right\rbrace &=& l(l+1), \label{radialequation} \\
\dfrac{1}{Y} \left\lbrace   \dfrac{1}{ \sin \theta}  \dfrac{d}{d \theta} \left( \sin \theta \dfrac{d Y}{d \theta} \right) + \dfrac{1}{\sin^2 \theta} \dfrac{d^2 Y}{d \varphi^2} \right\rbrace &=& -l(l+1). \label{angularequation}
\end{eqnarray}
As the potential $V(r)$ depends on $r$ only, there is no angular term. The solution for the angular equation \ref{angularequation} are the spherical harmonics. The normalized form are:
\begin{equation}
Y_l^m (\theta,\varphi) = \epsilon \sqrt{\dfrac{2l+1}{4\pi} \dfrac{(l-\vert m \vert )!  }{ (l+\vert m \vert )! }  } e^{im\varphi} P_l^m (\cos \theta)  \label{harmonicosesfericos}
\end{equation}
The graphical visualization of the first 4 spherical harmonics are depicted in the Figures \ref{HE_l0} to \ref{HE_l1_mm1}.
\begin{figure}[!h]
\centering
\subfigure{a)
\includegraphics[width=1.6 in]{figuras/HE_l0.pdf}
\label{HE_l0} 
}
\subfigure{b)
\includegraphics[width=1.6 in]{figuras/HE_l1_m0.pdf}
\label{HE_l1_m0}
} \\
\subfigure{c)
\includegraphics[width=2.0 in]{figuras/HE_l1_mm1.pdf}
\label{HE_l1_mm1}
}
\subfigure{d)
\includegraphics[width=2.0 in]{figuras/HE_l1_m1.pdf}
\label{HE_l1_m1}
} 
\caption{Graphical representation of the modulo $\vert Y_l^m \vert $ of the spherical harmonics $Y_l^m = \vert Y_l^m \vert e^{i\xi_l^m}$. The color is the complex phase $\xi_l^m$. \subref{HE_l0} $l=m=0$. \subref{HE_l1_m0} $l=1$ e $m=0$. \subref{HE_l1_mm1} $l=1$ e $m=-1$. \subref{HE_l1_m1} $l=1$ e $m=1$.}
\end{figure}


{\color{red}3b). Value:}
The potential $V(r)$ affects only the radial equation \ref{radialequation}. Making the substitution $R(r) = u(r)/r$ we have:
\begin{equation}
-\dfrac{\hbar^2}{2m} \dfrac{d^2u}{dr^2} + \left[ V(r) +  \dfrac{\hbar^2}{2m} \dfrac{l(l+1)}{r^2} \right] u = Eu. \nonumber
\end{equation}
Using the electrostatic potential $V(r)$ the radial equation becomes:
\begin{equation}
-\dfrac{\hbar^2}{2m} \dfrac{d^2u}{dr^2} + \left[  \dfrac{\hbar^2}{2m} \dfrac{l(l+1)}{r^2} -\dfrac{e^2}{4\pi \epsilon_0 r} \right] u = E u. \nonumber
\end{equation}

{\color{red}3c). Value:}
The first state corresponds to $n=1$ and $l=m=0$. So:
\begin{eqnarray}
\Psi_{100} = e^{-r/a} \sqrt{\left( \dfrac{2}{a}\right)^3 \dfrac{1}{2}} L_0^1 \left(  \dfrac{2r}{a} \right) Y_0^0(\theta,\phi) \nonumber  
\end{eqnarray}
From the tables we use $L_0^1 (x) = 1$ and $Y_0^0= 1/\sqrt{4\pi}$, so:
\begin{equation}
\Psi_{100} = e^{-r/a} \sqrt{ \dfrac{4}{a^3} } \dfrac{1}{\sqrt{4\pi}} = \dfrac{e^{-r/a}}{\sqrt{\pi a^3}}. \nonumber
\end{equation}
The first excited state is:
\begin{equation}
\Psi_{200} = e^{-r/2a} \left( \dfrac{2r}{2a}\right)^0 \sqrt{ \left( \dfrac{2}{2a}\right)^3  \dfrac{(2-1)!}{4(2!)^3} } L_{2-1}^1 \left( \dfrac{2r}{2a}\right) Y_0^0. \nonumber
\end{equation}
From the tables we have $L_1^1(x) = 4-2x$, so:
\begin{eqnarray}
\Psi_{200} &=& e^{-r/2a}  \sqrt{ \dfrac{1}{2^5a^3} } L_1^1 \left( \dfrac{r}{a}\right) Y_0^0, \nonumber \\
&=& e^{-r/2a}  \dfrac{1}{4\sqrt{2a^3}}  \left(4- 2\dfrac{r}{a}\right) \dfrac{1}{\sqrt{4\pi}} = \dfrac{e^{-r/2a}}{4 \sqrt{2\pi a^3} } \left(2- \dfrac{r}{a}\right)
\end{eqnarray}


\end{document}



