\documentclass[11pt,openany,oneside]{article}

\usepackage{makeidx}
\usepackage{latexsym}

%estes dois packages sao para tornar automaticos a acentuacao grafica em portugues: ã, é, ô, etc...
%infelizmente e misteriosamente, isso nao funciona no meu mac, no windows funcionava...
%tendo que usar \~a para ã, etc... 

\usepackage[latin1]{inputenc}
\usepackage[portuguese]{babel}
\usepackage[T1]{fontenc}
\usepackage{amsmath}
\usepackage{amsfonts}

% essa linha foi utilizada para corrigir o erro
% auto expansion is only possible with scalable
\usepackage{lmodern}

\usepackage{titlesec}

% possibilida criar figuras 1a) e 1b), por exemplo
\usepackage{subfigure}
\usepackage[pdftex]{graphicx}

%possibilita criar figuras com legenda na lateral
\usepackage{sidecap}

\usepackage[nottoc,notlot,notlof]{tocbibind}
\usepackage{capt-of}
\usepackage[left=4.0cm, right=2.0cm, top=4.0cm,bottom=3.0cm]{geometry}%normas icontec

%colocar cores nas fontes
\usepackage{color}
%\usepackage[usenames,dvipsnames]{color}

\usepackage{calc}
\usepackage{titletoc}

%para criar o espacamento usual no primeiro paragrafo
\usepackage{indentfirst}

\usepackage{pifont}

% para criar os links no texto quando cita equacoes, figuras, referencias, secoes, etc...
\usepackage{hyperref}

% para colorir os links criados anteriormente
\hypersetup{ colorlinks,
linkcolor=blue,
filecolor=darkgreen,
urlcolor=blue,
citecolor=blue }
\usepackage{microtype}
\usepackage[nottoc]{tocbibind} 
\usepackage{graphicx}
\usepackage{graphicx,color}
\usepackage{wrapfig}
\usepackage{epsfig}
\usepackage{amssymb}
\usepackage{amsthm}
\usepackage{verbatim}

% possibilita inserir paginas avulsas de arquivos pdf no meio do texto
\usepackage{pdfpages}

% altera a fonte nas legendas das figuras
\usepackage[font=small,format=plain,labelfont=bf,up,textfont=it,up]{caption}



% este define o espacamento entre as linhas
% para referencia, 1.5 'e um espacamento grande, utilizado na minha tese de doutorado
\renewcommand*{\baselinestretch}{1.2}

%comando para colocar vetor linha
%deve ser usado com o comando \vec{p}+\pvec{p}'=\pvec{p}''
\newcommand{\pvec}[1]{\vec{#1}\mkern2mu\vphantom{#1}}



\begin{document}

\hsize = 6.5 in
\hoffset = -0.5 in
\vsize = 9 in
\voffset = -0.5 in



\section*{{\color{red}Gabarito}}

\section*{Prova 1, 08/09/2014, Prof. Paulo Freitas Gomes}

\section*{Eletromagnetismo. Curso: F�sica}

\vspace{0.1 in}
\noindent
{\color{blue}1) Solu��o}: A trajet�ria do entre os limites � na dire��o perpendicular ao eixo do fio, logo $l=s$. Vamos supor que o referencial seja na posi��o $s=\Re =a$, de forma que $V(a) = 0$.  O campo  $\vec{E}$ do fio � perpendicular ao eixo do fio (eixo $z$), de forma que ele � paralelo ao vetor $d\vec{l} = d\vec{s}$. Sendo paralelo, temos que $\vec{E} \cdot d\vec{l} = Edl \cos \theta = Eds$, j� que $\theta = 0$. Usando tudo isso, temos que da defini��o de potencial:
\begin{equation}
V(s) = - \int_{\Re}^s \vec{E} \cdot d\vec{l} = - \int_a^s E ds = - \dfrac{\lambda}{2\pi \varepsilon_0}\int_a^s \dfrac{ds}{s} =  {\color{blue}\boxed{-\dfrac{\lambda}{2\pi \varepsilon_0 } (\ln s - \ln a)}} \nonumber
\end{equation}
 Usei tamb�m a integral de $\int_a^b \dfrac{dx}{x} = (\ln b - \ln a) $ dada no formul�rio da prova. Nesta express�o vemos por que n�o podemos assumir o referencial no infinito ($a=\infty$) ou na superf�cie do fio ($a=0$), pois em ambos os casos $\ln a$ n�o est� definido. {\color{red}2,5 pontos}.

\noindent
Para a segunda pergunta, vou calcular o gradiente de $V(s)$, usando a defini��o do vetor $\vec{\nabla}$ em coordenadas cil�ndricas dada na prova. Temos que:
\begin{equation}
 \dfrac{\partial V(s)}{\partial \phi} = \dfrac{\partial V(s)}{\partial z} = 0 \nonumber
\end{equation}
Logo:
\begin{eqnarray}
-\vec{\nabla} V(s) &=&  \hat{s} \dfrac{\partial}{\partial s} \left[ \dfrac{\lambda}{2\pi \varepsilon_0 } (\ln s - \ln a) \right]
=  \hat{s} \dfrac{\lambda}{2\pi \varepsilon_0 } \dfrac{\partial}{\partial s} (\ln s - \ln a) =  {\color{blue}\boxed{ \dfrac{\lambda}{2\pi \varepsilon_0 s}\hat{s}}} \nonumber 
\end{eqnarray}
O �ltimo termo � exatamente o campo dado no enunciado. Usei nessa dedu��o $\dfrac{\partial}{\partial s}(\ln s)  = \dfrac{1}{s}$. {\color{red}2,5 pontos}.

\_\_\_\_\_\_\_\_\_\_\_\_\_\_\_\_\_\_\_\_\_\_\_\_\_\_\_\_\_\_\_\_\_\_\_\_\_\_\_\_\_\_\_\_\_\_\_\_\_\_\_\_\_\_\_


\vspace{0.1 in}

\noindent
{\color{blue}2a) Solu��o}: O campo el�trico de uma esfera r�gida, s�lida, de raio $R$ e carga $q$, fora da esfera ($r>R$) � igual ao campo de uma carga pontual $q$:
\begin{equation}
{\color{blue}\boxed{\vec{E}_f = \dfrac{q}{4\pi \varepsilon_0 r^2}\hat{r}}} \label{campoforadaesfera}
\end{equation}
{\color{red}1 ponto.}

\noindent
{\color{blue}2b) Solu��o}: A densidade de energia armazenada no campo el�trico � $u = \frac{1}{2}\varepsilon_0 E^2$, onde $E=\vert \vec{E} \vert$. Assim, para encontrar a energia total $W$ devemos integrar a densidade de energia em todo o espa�o, tanto dentro quanto fora da esfera. O campo fora da esfera � dado pela eq. \ref{campoforadaesfera} e o campo $\vec{E}_d$ dentro da esfera  � dado no enunciado. Integrando nas coordenadas esf�ricas ($r,\theta,\phi$), teremos:
\begin{eqnarray}
W &=& \frac{1}{2}\varepsilon_0 \int_{\Omega} E^2 d\tau = \frac{1}{2}\varepsilon_0 \int_0^{2\pi} d\phi \int_0^{\pi} d\theta \int_0^{\infty} E^2 r^2\sin \theta dr d\theta d\phi \nonumber \\
 &=& \frac{1}{2} 2\pi \varepsilon_0 \int_0^{\pi} \sin \theta d\theta \left( \int_0^R E_d^2 r^2 dr + \int_R^{\infty} E_f^2 r^2 dr \right)  \nonumber \\
 &=& 2\pi \varepsilon_0\dfrac{q^2}{(4 \pi \varepsilon_0)^2} \left( \int_0^R \dfrac{r^2}{R^6} r^2 dr + \int_R^{\infty} \dfrac{r^2 dr}{r^4}  \right) = \dfrac{q^2}{8\pi\varepsilon_0} \left( \int_0^R \dfrac{r^4}{R^6} dr + \int_R^{\infty} \dfrac{dr}{r^2} \right) \nonumber \\
 &=&  \dfrac{q^2}{8\pi\varepsilon_0} \left( \dfrac{r^5}{5R^6}\biggr\rvert_0^R  - \dfrac{1}{r}\biggr\rvert_R^{\infty} \right) = \dfrac{q^2}{8\pi\varepsilon_0} \left( \dfrac{1}{5R}  + \dfrac{1}{R} \right)  =  {\color{blue}\boxed{\dfrac{1}{4\pi \varepsilon_0} \dfrac{3q^2}{5R}}}   \nonumber
\end{eqnarray} 
Quanto maior o raio $R$, menor a energia, pois maior � a dist�ncia final entre os elementos de carga que formam $q$. A integral em $\phi$ � direta e em $\theta$ � igual a 2, pois $\int \sin \theta d\theta = - \cos \theta$. Os campos dependem apenas de $r$, com express�es diferentes para dentro ($r<R$) e fora ($r>R$) da esfera. {\color{red}4 pontos.}

 
\end{document}



