\documentclass[11pt,openany,oneside]{article}

\usepackage{makeidx}
\usepackage{latexsym}

%estes dois packages sao para tornar automaticos a acentuacao grafica em portugues: ã, é, ô, etc...
%infelizmente e misteriosamente, isso nao funciona no meu mac, no windows funcionava...
%tendo que usar \~a para ã, etc... 

\usepackage[latin1]{inputenc}
\usepackage[portuguese]{babel}
\usepackage[T1]{fontenc}
\usepackage{amsmath}
\usepackage{amsfonts}

% essa linha foi utilizada para corrigir o erro
% auto expansion is only possible with scalable
\usepackage{lmodern}

\usepackage{titlesec}

% possibilida criar figuras 1a) e 1b), por exemplo
\usepackage{subfigure}
\usepackage[pdftex]{graphicx}

%possibilita criar figuras com legenda na lateral
\usepackage{sidecap}

\usepackage[nottoc,notlot,notlof]{tocbibind}
\usepackage{capt-of}
\usepackage[left=4.0cm, right=2.0cm, top=4.0cm,bottom=3.0cm]{geometry}%normas icontec
\usepackage{color}
\usepackage{calc}
\usepackage{titletoc}

%para criar o espacamento usual no primeiro paragrafo
\usepackage{indentfirst}

\usepackage{pifont}

% para criar os links no texto quando cita equacoes, figuras, referencias, secoes, etc...
\usepackage{hyperref}

% para colorir os links criados anteriormente
\hypersetup{ colorlinks,
linkcolor=blue,
filecolor=darkgreen,
urlcolor=blue,
citecolor=blue }
\usepackage{microtype}
\usepackage[nottoc]{tocbibind} 
\usepackage{graphicx}
\usepackage{graphicx,color}
\usepackage{wrapfig}
\usepackage{epsfig}
\usepackage{amssymb}
\usepackage{amsthm}
\usepackage{verbatim}

% possibilita inserir paginas avulsas de arquivos pdf no meio do texto
\usepackage{pdfpages}

% altera a fonte nas legendas das figuras
\usepackage[font=small,format=plain,labelfont=bf,up,textfont=it,up]{caption}



% este define o espacamento entre as linhas
% para referencia, 1.5 'e um espacamento grande, utilizado na minha tese de doutorado
\renewcommand*{\baselinestretch}{1.2}

%comando para colocar vetor linha
%deve ser usado com o comando \vec{p}+\pvec{p}'=\pvec{p}''
\newcommand{\pvec}[1]{\vec{#1}\mkern2mu\vphantom{#1}}



\begin{document}

\noindent

\section*{Prova 1, 08/09/2014, Prof. Paulo Freitas Gomes}

\section*{Eletromagnetismo. Curso: F�sica}

\vspace{0.3 in}

Nome: \_\_\_\_\_\_\_\_\_\_\_\_\_\_\_\_\_\_\_\_\_\_\_\_\_\_\_\_\_\_ Matr�cula:\_\_\_\_\_\_\_\_

\vspace{0.1 in}

Dados: $\varepsilon_0 = 8,85\times 10^{-12}$ C$^2$N$^{-1}$m$^{-2}$, carga elementar do el�tron $-e= -1,6\times 10^{-19}$ C, volume da esfera $\dfrac{4}{3}\pi r^3$, �rea da esfera $4\pi r^2$. Use $\tau$ para o volume. %Escolha apenas 2 problemas para resolver.

\vspace{0.1 in}

1) Seja um fio longo reto infinito com uma densidade linear de carga $\lambda$. O campo el�trico gerado por este fio �:
\begin{equation}
\vec{E} = \dfrac{1}{4\pi \varepsilon_0} \dfrac{2\lambda}{s} \hat{s} \nonumber
\end{equation}
Encontre o potencial $V(s)$ gerado pelo fio. Mostre que seu potencial satisfaz: $\vec{E} = - \vec{\nabla} V(s)$. As coordenadas cil�ndricas s�o ($s,\phi,z$) tais que $s^2 = x^2 + y^2$ e $\tan \phi = \dfrac{y}{x}$. 

\vspace{0.1 in}

2) a) Seja uma esfera r�gida s�lida uniformemente carregada de raio $R$ e carga $q$. Qual o campo el�trico fora da esfera ($r>R$)? b) Encontre a energia desta esfera que est� armazenada no campo el�trico. Ajuda: dentro da esfera ($r<R$) o campo �:
\begin{equation}
\vec{E} = \frac{q}{4\pi \varepsilon_0} \dfrac{r}{R^3}\hat{r} \nonumber
\end{equation}

%3) Encontre o campo el�trico dentro de uma esfera que cont�m uma densidade de carga proporcional a dist�ncia da origem: $\rho = kr$, onde $k$ � uma constante. Dica: esta densidade n�o � constante e vc deve integr�-la para obter a carga total. 

 \subsection*{F�rmulas para consulta}
 
 
\begin{equation}
V(s) = - \int_{\Re}^s \vec{E} \cdot d\vec{l} \quad \quad V(\Re) = 0 \quad \quad \int_a^b \dfrac{dx}{x} = \ln b - \ln a = \ln \dfrac{b}{a} \quad \quad   W = \frac{1}{2}\varepsilon_0 \int_{\Omega} E� d\tau   \nonumber 
\end{equation} 
%\vspace{0.01 in} 
 
\begin{equation}
 \quad \quad d\tau = dxdydz = sdsd\phi dz = r^2 dr \sin \theta d\theta d\phi \quad \quad \int_a^b x^n dx = \dfrac{b^{n+1}-a^{n+1}}{n+1} \nonumber 
\end{equation} 
%\vspace{0.01 in}

\begin{equation}
\vec{F} = \dfrac{Q_1Q_2}{4\pi \varepsilon_0 r^2}\hat{r} \quad \quad \vec{\nabla} \cdot \vec{E} = \dfrac{\rho}{\varepsilon_0} \quad \quad  \int_S \vec{E} \cdot d\vec{A} = \dfrac{Q_S}{\varepsilon_0}  \quad \quad W = \frac{1}{2} \int \rho V d\tau  \quad \quad \vec{\nabla} \times \vec{E} = 0 \nonumber 
\end{equation} 

%\vspace{0.01 in}

\begin{equation}
 \vec{\nabla} = \hat{x}\dfrac{\partial}{\partial x}+\hat{y}\dfrac{\partial}{\partial y}+\hat{z}\dfrac{\partial}{\partial z}   =  \hat{s}\dfrac{\partial}{\partial s}+\dfrac{\hat{\phi}}{s}\dfrac{\partial}{\partial \phi}+\hat{z}\dfrac{\partial}{\partial z} \quad \quad Q = \int_{\Omega} \rho (r) d\tau \nonumber 
\end{equation} 



 
\end{document}



